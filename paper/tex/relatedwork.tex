\section{Related Work}\label{sec:relatedwork}
Many research studies have been using smartphones as the testbed or platform to evaluate other information such as environmental information. For example, \cite{lukac2011soundproof} uses smartphone as the platform to monitor wildlife and environment. \cite{jo2013towards} uses the smartphones to build a motion recognition testbed. \cite{heo2010user} analyzes the user demand on the smartphones through a virtual phone environment. However, ReCapture is different from those research because ReCapture is concentrated on the testbed for smartphone related system evaluation. It provides a general testbed for the developers to validate the ideas which could potentially improve the performance on the smartphones or to collect data and study the behaviors in a large scale flavor.

On the other hand, the virtualization technology has improved a lot in the data center which provides the core technology for cloud computing. The virtualization techniques are extremely challenging in mobile environment because of the hardware limitations of the smartphones. \cite{andrus2011cells} implements a virtual mobile phone architecture that could run several virtual phones on the same physical devices. \cite{acharya2009phone} is another approach to achieve the virtualization through the microkernel hypervisor. ReCapture relies on these virtualization technology to scale up the testbed within the limited number of smartphones.
