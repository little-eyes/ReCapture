\section{Introduction}\label{sec:introduction}
Mobile devices have been more and more popular in commercial products which gives a user incredible convenience to access Internet services and search the local information. On the other hand, people in research field also realize the significance a mobile device can bring to them. Therefore, more and more research topics have been studied within the smartphones.

For example, researchers are caring about the energy efficiency of a mobile devices like a smartphone or a tablet. They realize the problem is multi-layered which indicates the operating system, network protocol or hardware could have problem. Many literatures have been trying to address this challenges by proposing some techniques and implementing a prototype. Unfortunately, it is difficult for others to repeat the work as they did and get the same results. In other words, the methodology here to evaluate the system is not a complete and reasonable way. Most of them follow a pattern that evaluate the system's performance under one or two smartphones, then using an existing data set to simulate what will happen in practice. However, the simulation itself cannot prove any important conclusion.

The major reason causing this type of incomplete methodology is the reality that we do not have enough devices, time and human force to conduct a large scale experiment. Thereafter, we still have no idea how an idea could work in practice. Even if we have above restriction satisfied, we encounter another problem that it is very hard to control an experiment if we want a person to be involved. For instance, if we want to study the smartphone performance relationship between the user's usage behavior. We may ask the participants to follow some instructions during the experiment. Unfortunately, it is very hard to control the errors between the instructions and the actual use. This could lead to terrible misconclusions in social studies.

In this paper, we propose \emph{ReCapture} which is a large scale automatic testbed for mobile devices. ReCapture is capable of recreating the usage reality from a standard usage trace and a developer is able to plugin any data collection modules to retrieve the desired data. In case that limited physical devices are available, the ReCapture is able to use the virtualization technology to scale up the amount of devices performing an experiments concurrently. 

For a specific experiment, the researcher needs to predefine some user activity log for each devices. Fortunately, many existing data set about a user's usage trace can be the perfect activity log. Those activity logs will trigger each device to follow the same usage behavior on the smartphone. During each usage phase, for instance a user is using Facebook app, the testbed will issue the corresponding screen activity which automatically operate the phone until the usage phase is switched to another app based on the activity log. In this way, a developer can deploy large scale system evaluation based on real user's records instead of simulating the user behavior or relying on the profiling techniques to approximate some metrics like energy consumption.

The rest of the paper is arranged in the following way: Section~\ref{sec:relatedwork} discusses some related work regarding the testbed and virtualization technology in mobile system; Section~\ref{sec:design} describes the system design details and Secion~\ref{sec:implementation} illustrates the system's implementation details; then Section~\ref{sec:evaluation} provides some preliminary performance; finally Section~\ref{sec:conclusion} is our conclusion.
