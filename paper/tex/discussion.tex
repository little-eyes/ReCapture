\section{Discussion and Future Work}\label{sec:discussion}
In this section, we discuss some important factors in the system which could be improved through different optimizations.

The first one we are interested in is the system's architecture. As is illustrated in the design section, we separate the system into two parts: on-device part and central master. However, it is possible to move the scheduler into the central master so that the on-device component only contains the performance monitor module. This design makes the architecture on-device very slim and lightweight. We did not design in this way for two reasons: 1) if the testbed is in large scale, or we have the virtual machines running in the device, centralize the scheduler for all devices will increase the delay a lot. Even though we did not do the relevant experiment, putting thousands of devices' scheduling information into a single place is not an easy task; 2) the distributed design could minimize the cost of single point failure. It is possible the central master could crash, if we put all the scheduling information in the central master, we have to re-do the experiment again if we do not want to lose any information. In contrast, if the scheduler is distributed, the failure cost is only a small portion. Restart the central master, then everything is still on-schedule.

Another issue we think important is the virtualization technology. In Section~\ref{sec:relatedwork} we have discussed some existing work of virtualizing smartphones. Unfortunately, the scalability of virtualization on smartphone is not satisfactory. Usually, two or three virtual phones in one physical phones have reached the limits. In our case, we are caring scalability more than any other factors. Therefore, the existing technology may not be a good fit.

The new approach to make this happen, we believe, could be something more lightweight and should be in higher level, say application level. We think the application itself could be virtualized. First, if we virtualize applications, we need the possibility that two or more apps could run simultaneously while we only have one screen. This problem has been handled very well in Windows 8 operating system which you can run two apps side by side, but we still need more research in the Android system. Second, application vitrualization does not require a whole operating system stack to be virtualized, in which way the extra cost could be minimized.
